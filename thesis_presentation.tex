\documentclass[18pt]{beamer}

% Add some reduced spacing itemize enviroments
\usepackage{paralist}

% Increase spacing in lists
\newlength{\wideitemsep}
\setlength{\wideitemsep}{\itemsep}
\addtolength{\wideitemsep}{5pt}
\let\olditem\item
\renewcommand{\item}{\setlength{\itemsep}{\wideitemsep}\olditem}


% Hacky fix to bug in beamer
\let\beameritemnestingprefix\hbox{}


% University of Manchester colours
\DefineNamedColor{named}{uompurple}{rgb}{0.427,0,0.616}
\usecolortheme[named={uompurple}]{structure}

% Theme (for shapes etc.)
\usetheme{Madrid}

% remove navigation symbols
\setbeamertemplate{navigation symbols}{}

% Italic captions
\usepackage[textfont=it]{caption}

% small footnotes
\usepackage{setspace}



\title[??ds]{??ds}

\author[David Shepherd]{David Shepherd}

\institute[UoM]{The University of Manchester}



\begin{document}

\begin{frame}
  \titlepage
\end{frame}

\begin{frame}{How good is GI}

  \begin{itemize}
  \item Plenty of work on coming up with GI methods for micromagnetics
  \item GI integration said to be very useful in other areas (e.g. ??ds)
  \item Plenty of claims that GI is important
  \item But no numerical evidence! (i.e. comparisons of overall accuracy of
    GI methods vs normal).
    % ??ds check gI on spheres paper
  \end{itemize}

\end{frame}

\begin{frame}{How good is GI}

  So:
  \begin{itemize}
  \item Experimented with IMR vs TR vs BDF2 for two problems with
    known analytical solutions.
  \item Found that TR and IMR are very similar, but BDF2 is quite a bit worse
  \item However, due to some unique properties of these analytical
    solutions TR had some GI properties at times...
  \item Unfortunately no time to experiment with comparisons for full MM
    problem with magnetostatics (since full IMR GI method not finished).
  \end{itemize}
  
\end{frame}


\begin{frame}{Making IMR (GI) more efficent}

  \begin{itemize}
  \item Quite a few papers on IMR in micromagnetics
  \item But:
    \begin{itemize}
    \item using simple and inefficent non-linear solvers
    \item no adaptivity
    \item Often idealised problems (no magnetostatics, simple cubeoids)
    \end{itemize}
  \end{itemize}
\end{frame}

\begin{frame}{Making IMR (GI) more efficent}

  So:
  \begin{itemize}
  \item Created better non-linear solver (incl. internal linear solver)
    % ??ds overstating it?
  \item Created adaptive IMR algorithm
  \end{itemize} 
  
\end{frame}


\begin{frame}{Do we need implicit methods}

  \begin{itemize}
  \item Little research about when implicit methods are needed in micromagnetics
  \item Some papers say that implicit methods are much more efficent
  \item But many people use explicit methods without compliants
  \item Stiffness only studied for macrospin models and mumag4 problem
  \item Some mixed ideas about where stiffness comes from (precession vs
    damping timescales? exchange vs magnetostatics propogation timescales?
    discretisation?) (not in publications but my impression from talking
    to people)
  \end{itemize}

\end{frame}

\begin{frame}{Do we need implicit methods }

  So:
  \begin{itemize}
  \item Tried to study more general origins of stiffness.
  \item Found that you can explain observed stiffness with discretisation
    alone.
  \item Characterised discretisation lengths where stiffness occurs (for
    FEM and FEM/BEM methods).
  \end{itemize} 
  
\end{frame}

\end{document}

%%% Local Variables:
%%% mode: latex
%%% TeX-master: true
%%% End:
