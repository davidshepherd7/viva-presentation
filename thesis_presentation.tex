\documentclass[18pt]{beamer}

% Add some reduced spacing itemize enviroments
\usepackage{paralist}

% Increase spacing in lists
\newlength{\wideitemsep}
\setlength{\wideitemsep}{\itemsep}
\addtolength{\wideitemsep}{5pt}
\let\olditem\item
\renewcommand{\item}{\setlength{\itemsep}{\wideitemsep}\olditem}


% Hacky fix to bug in beamer
\let\beameritemnestingprefix\hbox{}


% University of Manchester colours
\DefineNamedColor{named}{uompurple}{rgb}{0.427,0,0.616}
\usecolortheme[named={uompurple}]{structure}

% Theme (for shapes etc.)
\usetheme{Madrid}

% remove navigation symbols
\setbeamertemplate{navigation symbols}{}

% Italic captions
\usepackage[textfont=it]{caption}

% small footnotes
\usepackage{setspace}



\title{Numerical methods for micromagnetics}

\author[David Shepherd]{David Shepherd}

\institute[UoM]{The University of Manchester}

\date{2nd February 2015}


\begin{document}

\begin{frame}
  \titlepage
\end{frame}

\begin{frame}{Overall goal}
  \begin{center}
    {\huge Goal: Use better numerical methods to speed up calculations.}
  \end{center}

  \begin{itemize}
  \item Create new (better) numerical methods
    % obviously the main way to do this is to come up with new numerical
    % methods.
  \item Find out when a given method is better/worse
    % but also, A lot of the time certain methods will be better for
    % certain situations but worse for others. We need to know when to use
    % a method as well.
  \end{itemize}
\end{frame}

\begin{frame}{Overview}
  \begin{itemize}
  \item I focused on methods that use \emph{implicit midpoint rule} for time discretisation.
  % IMR can conserve important properties of the model, hope
  % will give better overall accuracy + also it's stable.
  \item Found improvements to existing IMR-based numerical methods.
  \item Characterised when it is beneficial to use IMR.
  \end{itemize}
\end{frame}


\begin{frame}{Making IMR more efficient}

  \begin{itemize}
  \item Quite a few papers on IMR in micromagnetics
  \item But:
    \begin{itemize}
    \item no adaptivity
    \item using simple and inefficient non-linear solvers
    \end{itemize}
 \end{itemize}
 \end{frame}

\begin{frame}{Making IMR more efficient}

  So:
  \begin{itemize}
  \item Created adaptive IMR algorithm
  \item Created better solver for the discrete problem [caveat: need an
    effective preconditioner for LLG block].
  \end{itemize}
  % these are the main contributions of my thesis

\end{frame}



\begin{frame}{When do we need implicit (stable) methods?}

  \begin{itemize}
  \item Little research about when implicit methods are needed in micromagnetics (a.k.a. stiffness)
  \item Some papers say that implicit methods are much more efficient, but others use explicit methods without complaints
  \item Some mixed ideas about where stiffness comes from
  % (precession vs damping timescales? exchange vs magnetostatics propagation
  % timescales?
  % discretisation?) (not in publications but my impression from talking
  % to people)
  \end{itemize}

\end{frame}

\begin{frame}{When do we need implicit (stable) methods?}

  So:
  \begin{itemize}
  \item Studied origin of stiffness in micromagnetics.
  \item Can explain stiffness with spatial discretisation alone.
  \item Characterised discretisation lengths where stiffness occurs (for
    FEM and FEM/BEM methods).
  \end{itemize}

\end{frame}



\begin{frame}{How useful is geometric integration?}

  \begin{itemize}
  \item Plenty of papers on geometric integration methods for micromagnetics
  \item GI said to be very useful in other areas
  \item Plenty of claims that GI is important
  \item But little numerical evidence (\textit{i.e.}\ comparisons of errors)!
  \end{itemize}

\end{frame}

\begin{frame}{How useful is geometric integration?}

  So:
  \begin{itemize}
  \item Experimented with a IMR, trapezoid rule and BDF2 for two problems with
    analytical solutions.
  \item Found TR and IMR perform similarly, but BDF2 is quite a bit worse.
  \item However, for these examples, TR seems to have some GI properties.
    % Both non-dissipative, for sufficiently simple cases they are the same
    % method, maybe TR with another method of conserving |m| would be GI?
  \end{itemize}

\end{frame}



\begin{frame}{Summary}
  \begin{itemize}
  \item Created adaptive IMR method
  \item Created efficient solver
  \item Characterised when stiffness is an issue in FEM/BEM simulations
  \item No strong conclusions possible on overall usefulness of GI
  \end{itemize}
\end{frame}

\end{document}

%%% Local Variables:
%%% mode: latex
%%% TeX-master: true
%%% End:
